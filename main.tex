\documentclass[a4paper]{scrartcl}
\pagestyle{plain}
\usepackage{a4wide}
\usepackage[ngerman]{babel}
\usepackage{tikz}
\usepackage{enumerate}
\usepackage{listings}
\usepackage[colorlinks=true,linkcolor=black]{hyperref}
\usetikzlibrary{calc}
\usetikzlibrary{matrix}
\usetikzlibrary{automata}
\usetikzlibrary{decorations.pathreplacing}
\usetikzlibrary{positioning}
\usepackage{dsfont}
\usepackage{amsmath, amsthm, amssymb}
\usepackage{algorithm}
\usepackage{algpseudocode}
\usepackage{color}
\usepackage{scrpage2}
\usepackage[T1]{fontenc}
\usepackage[utf8]{inputenc}
%\usepackage[math,light]{anttor}
\usepackage{textcomp}
\usepackage{etoolbox}

% Pretty captions!
%\usepackage{caption}
%\DeclareCaptionFont{white}{\color{white}}
%\DeclareCaptionFormat{listing}{\colorbox{gray}{\parbox{0.98\linewidth}{#1#2#3}}}
%\captionsetup[lstlisting]{format=listing,labelfont=white,textfont=white}

% Pretty page headings!
\pagestyle{scrheadings}
\clearscrheadings
\clearscrplain
\setfootwidth{head}
\ofoot{\pagemark}

\makeatletter
% Different format for headings
\def\section{\@startsection{section}{1}%
  \z@{.7\baselineskip\@plus\baselineskip}{.5\baselineskip}%
  {\normalfont\Large\scshape\centering}}
% This does spacing around caption.
\setlength{\abovecaptionskip}{0.5em}
\setlength{\belowcaptionskip}{0.5em}
% This does justification (left) of caption.
\long\def\@makecaption#1#2{%
  \vskip\abovecaptionskip
  \sbox\@tempboxa{#2}%
  \ifdim \wd\@tempboxa >\hsize
    #2\par
  \else
    \global \@minipagefalse
    \hb@xt@\hsize{\box\@tempboxa\hfill}%
  \fi
  \vskip\belowcaptionskip}
\providerobustcmd*{\bigcupdot}{%
  \mathop{%
    \mathpalette\bigop@dot\bigcup
  }%
}
\newrobustcmd*{\bigop@dot}[2]{%
  \setbox0=\hbox{$\m@th#1#2$}%
  \vbox{%
    \lineskiplimit=\maxdimen
    \lineskip=-0.7\dimexpr\ht0+\dp0\relax
    \ialign{%
      \hfil##\hfil\cr
      $\m@th\cdot$\cr
      \box0\cr
    }%
  }%
}
\makeatother

\setkomafont{title}{\normalfont}
\setkomafont{pageheadfoot}{\normalfont}
\setkomafont{pagenumber}{\normalfont\scshape}
\setkomafont{disposition}{\normalfont\bfseries}

\widowpenalty=300
\clubpenalty=300

\newtheorem*{defn}{Definition}
\newtheorem*{exmp}{Beispiel}
\newtheorem*{satz}{Satz}
\newtheorem*{lemm}{Lemma}
\newtheorem*{korr}{Korollar}

\renewcommand{\algorithmicrequire}{\textbf{Input:}}
\renewcommand{\algorithmicensure}{\textbf{Output:}}

\title{Randomisierte Algorithmen\\
       Sommersemester 2011 \\
       \small Mitschrift der Tafelanschrift von Prof. Wanka}

\begin{document}
\ohead{5. Mai 2011}
\section{Beispiele (\glqq zum Warmwerden\grqq)}
\subsection{Randomized Quicksort}
\begin{algorithm}
	\caption{Algorithmus randQS ($S$: Array aus $n$ verschiedenen Zahlen)}
	\vspace{0.4cm}
	\begin{enumerate}
		\setlength{\itemsep}{2pt}
		\setlength{\parskip}{2pt}
		\setlength{\parsep}{2pt}
		\item Wähle ein $y$ aus $S$ ($y$ heißt Pivotelement) u.a.r.\ (\textit{uniformly at random})
		\item $S_1 := \{ x \in S\ |\ x < y \} \quad S_2 := \{ x \in S\ |\ x > y \}$
		\item \textbf{if} $S_1 \not= \emptyset$ \textbf{then} randQS($S_1$)
		\item print $y$
		\item \textbf{if} $S_2 \not= \emptyset$ \textbf{then} randQS($S_2$)
	\end{enumerate}
\end{algorithm}
Wir messen die Laufzeit in der Anzahl der Vergleiche auf Schlüsseln.
\subsubsection{\textit{worst-case}-Szenario}
\begin{center}
\renewcommand\arraystretch{1.3}
\begin{tabular}{ccccccc|c|c}
	&&&&&&& \# Vergleiche & Wahrscheinlichkeit \\
	$1$&$2$&$3$&$4$&$5$&$6$&$7$&$6$&$\frac{1}{7}$\\
	&$2$&$3$&$4$&$5$&$6$&$7$&$5$&$\frac{1}{6}$\\
	&&&$\ddots$&&&&&\\
	&&&&&$6$&$7$&$1$&$\frac{1}{2}$\\
	&&&&&&$7$&$0$&$1$
\end{tabular}
\renewcommand\arraystretch{2.5}
\begin{tabular}{rl}
	Gesamtzahl der Vergleiche: & $\displaystyle \sum_{i=1}^{n-1} i = \frac{n(n-1)}{2} \in \Theta(n^2)$ \\
	Gesamtwahrscheinlichkeit: & $\displaystyle \prod_{i=1}^{n} \frac{1}{i} = \frac{1}{n!}$
\end{tabular}
\end{center}
\subsubsection{Probabilistische Laufzeitanalyse}
Wir bestimmen die erwartete Laufzeit: Sei $s_i$ der Schlüssel mit Rang $i$
($i$-t-kleinster Schlüssel). Wir definieren für $1 \leq i < j \leq n$ folgende
Zufallsvariable:
\[
  X_{ij} = \begin{cases} 1 & \text{falls der Algor. $s_i$ und $s_j$ vergleicht} \\
	  0 & \text{sonst} \end{cases}
\]
Die Gesamtzahl der Vergleiche ist $\sum_{i=1}^n \sum_{j=i+1}^n X_{ij}$.
Definiere $p_{ij} = P\left[X_{ij} = 1\right]$. Der Erwartungswert ist damit
wegen $E\left[X_{ij}\right] = 0 \cdot P\left[X_{ij} = 0\right] + 1 \cdot
P\left[X_{ij} = 1\right]$ gleich
\[
  E\left[\sum_{i=1}^n \sum_{j=i+1}^n X_{ij}\right] = \sum_{i=1}^n
  \sum_{j=i+1}^n E\left[X_{ij}\right] = \sum_{i<j} p_{ij}.
\]

Anhand folgenden Beispiels soll eine äquivalente Formulierung für $X_{ij}$
gezeigt werden (fettgedruckte Zahlen sind die in diesem Schritt neu gewählten
Pivotelemente, unterstrichene Zahlen entsprechen den Pivotelementen der
vorhergehenden Schritte):

\vspace{0.5cm}
\begin{minipage}{0.6\linewidth}
\renewcommand\arraystretch{1.3}
\centering
\begin{tabular}{ccccccc}
	$\mathbf{3}$ & $5$ & $2$ & $1$ & $5$ & $7$ & $6$ \\
	$2$ & $\mathbf{1}$ & $\underline{3}$ & $4$ & $\mathbf{5}$ & $7$ & $6$ \\
	$\underline{1}$ & $\mathbf{2}$ & $\underline{3}$ & $\mathbf{4}$ & $\underline{5}$ & $7$ & $\mathbf{6}$ \\
	$\underline{1}$ & $\underline{2}$ & $\underline{3}$ & $\underline{4}$ & $\underline{5}$ & $\underline{6}$ & $\mathbf{7}$
\end{tabular}
\end{minipage}\hfill\begin{minipage}{0.3\linewidth}
\begin{tikzpicture}[scale=0.7,level distance=10mm,level/.style={sibling distance=20mm/#1}]
	\tikzstyle{every node}=[draw,scale=0.7]
	\node (a){$3$}
	child {node (b) {$1$} child { node (c) {$2$} }}
	child {node (d) {$5$} child { node (e) {$4$} } child { node (f) {$6$} child { node (g) {$7$} } } };
\end{tikzpicture}
\end{minipage}
\vspace{0.5cm}

Die Schlüssel $s_i$ und $s_j$ werden also genau dann miteinander verglichen,
wenn nur $s_i$ oder $s_j$ als erstes aus $S_{ij} = \{ s_i, \dots, s_j \}$ als
Pivotelement gewählt werden:
\[
  X_{ij} = \begin{cases} 1 & \text{falls $s_j$-Knoten Nachfolger des
  $s_i$-Knoten oder umgekehrt} \\ 0 & \text{sonst} \end{cases}
\]
\begin{align*}
  \Rightarrow p_{ij} &= P\left[s_i \text{ wird als erstes Pivotelement aus $S_{ij}$ gewählt}\right] \\
  &+ P\left[s_j \text{ wird als erstes Pivotelement aus $S_{ij}$ gewählt}\right] = \frac{2}{j-i+1}
\end{align*}
Unter Verwendung der Eigenschaft der harmonischen Reihe $\ln(i+1) \leq H_i =
\sum_{k=1}^i \frac{1}{k} \leq 1 + \ln i$ ergibt sich für den Erwartungswert
\begin{align*}
	\sum_{i<j} p_{ij} &= \sum_{i=1}^n \sum_{j=1}^n \frac{2}{j-i+1} = 2 \cdot \sum_{i=2}^n \sum_{j=2}^i \frac{1}{j} \\
			  &\leq 2 \cdot \sum_{i=1}^n \left(H_i - 1\right) \leq 2 n \ln n = (2 \ln 2) n \log_2 n \approx 1.386\dots n \log_2 n
\end{align*}
\renewcommand\arraystretch{1.5}
\begin{tabular}{r|ccccccc}
	& & $j = 2$ & 3 & 4 & $\cdots$ & $n-1$ & $n$ \\
	\hline
	$i = 1$ & $0$ & $\frac{2}{2}$ & $\frac{2}{3}$ & $\frac{2}{4}$ & $\cdots$ & $\frac{2}{n-1}$ & $\frac{2}{n}$ \\
	$2$ && $0$ & $\frac{2}{2}$ & $\frac{2}{3}$ & $\cdots$ & $\cdots$ & $\frac{2}{n-1}$ \\
	$3$ &&& $0$ & $\frac{2}{2}$ & $\cdots$ & $\cdots$ & $\frac{2}{n-2}$
\end{tabular}

\section{Wahrscheinlichkeitsrechnung}
\subsection{Bälle und Kisten}
Szenario: $m$ Bälle und $n$ Kisten, ein Ball landet mit Wahrscheinlichkeit
$\frac{1}{n}$ in Kiste $i$, für alle $1\leq i\leq n$. $L_i$ bezeichne die
Anzahl der Bälle in Kiste $i$ (Last).

Wir interessieren uns für die maximale Last
\[
  L = \max\left\{ L_i\ |\ i \in \{1,\dots,n\}\right\},
\]
d.h. die Anzahl der Bälle in der vollsten Kiste.
\[
  \E[L_i] = 1, \quad \text{für } n \geq 2 \quad \E[L] > 1
\]

Beachte: es gibt einen Unterschied zwischen der erwarteten maximalen Last und
der maximalen erwarteten Last!

\begin{defn}
	Sei $X$ eine Zufallsvariable für ein Zufallsexperiment mit Parameter
	$n$. Sei $f : \mathbb{N} \mapsto \mathbb{R}^{\geq 0}$ eine Funktion.
	Eine Schranke der Form $X = \mathcal{O}\left(f(n)\right)$ gilt
	\emph{m.h.W.} (mit hoher Wahrscheinlichkeit), wenn gilt:
	\[ \forall\,d > 0 \ \exists\,c > 0 \ \exists\,n_0 \geq 1 \ \forall\,n
	\geq n_0 : \Pr\left[ X \geq c \cdot f(n)\right] \leq \frac{1}{n^d} \]
\end{defn}
\begin{satz}
	Bei $n$ Bällen und $n$ Kisten gilt:
	\[ L = \mathcal{O}\left(\frac{\log n}{\log \log n}\right) \ \
	\text{m.h.W.} \]
\end{satz}
\begin{proof}
	Betrachte feste Kiste $i$: $L_i \geq k$ gilt genau dann, wenn es eine
	Teilmenge $J \subseteq \{1, \dots, n \}$ der Bälle mit $|J| = k$ gibt,
	sodass alle Bälle aus $J$ in Kiste $i$ fallen.

	Für Ball $i$ und Kiste $j$ sei
	\[
	  X_i^j = \begin{cases} 1 & \text{falls Ball $i$ in Kiste $j$ landet}\\
		  0 & \text{sonst} \end{cases}
	\]

	\begin{align*}
	  \Pr[L_i \geq k] &= \Pr\left[ \exists\,J \subseteq \{1, \dots, n\}, |J| =
	  k : \forall\,j \in J : X_i^j = 1 \right] \\
	  &\leq \sum_{\substack{J\subseteq\{1,\dots,n\}\\|J|=k}}
	  \Pr\left[\forall\,j \in J : X_i^j = 1\right] \\
	  &= \sum_{\substack{J\subseteq\{1,\dots,n\}\\|J|=k}}
	  \left(\frac{1}{n}\right)^k = \binom{n}{k} \cdot \left(\frac{1}{n}\right)^k \\
	  \intertext{Mit $\left(\frac{n}{k}\right)^k \leq \binom{n}{k} \leq
	  \left(\frac{e\cdot n}{k}\right)^k$ gilt nun}
	  &\leq \left( \frac{e\cdot n}{k} \right)^k \cdot \left( \frac{1}{n}
	  \right)^k = \left( \frac{e}{k} \right)^k
	\end{align*}

	Es gilt nun noch zu zeigen, dass für jedes feste $d > 0$ ein $k =
	\mathcal{O}\left(\frac{\log n}{\log \log n}\right)$ existiert mit
	$\Pr[L_i \geq k] \leq \frac{1}{n^d}$:
	\[
	  k = \max \left\{ 2 d \cdot \frac{\log n}{\log \log n}, e \cdot
	  \sqrt{\log n} \right\} = \mathcal{O}\left(\frac{\log n}{\log \log
	  n}\right).
	\]
	Der Logarithmus sei o.B.d.A.\ zur Basis $2$ gewählt. Mit dieser Wahl
	von $k$ ergibt sich aus der vorherigen Abschätzung
	\begin{align*}
		\Pr[L_i \geq k] &\leq \left(\frac{e}{k}\right)^k \leq \left(
		\frac{1}{\sqrt{\log n}} \right)^{\frac{2\,d\,\log n}{\log \log
		n}} \\
		&= \left( \frac{1}{\log n} \right)^{\frac{d\,\log
		n}{\log \log n}} = \left(\frac{1}{2}\right)^{d\,\log n} \\
		&= n^{d\,\log \frac{1}{2}} = \frac{1}{n^d}
	\end{align*}
	
	Damit gilt für $L$:
	\[
	  \Pr[L \geq k] = \Pr\left[ \exists\,i : L_i \geq k \right] \leq
	  \sum_{i=1}^n \Pr[L_i \geq k] \leq n \cdot \frac{1}{n^d} =
	  \frac{1}{n^{d-1}}
	\]

	Ersetzt man in der gesamten Rechnung $d$ durch $d+1$ so ergibt sich die
	Behauptung.
\end{proof}
\subsection{Chernoff-Schranken}
\ohead{26. Mai 2011}
Im Bereich der randomisierten Algorithmen kommt es oft vor, dass wir
Zufallsvariable $X$ untersuchen, die die Summe von $0$-$1$-Zufallsvariablen
sind. Beispielsweise hatten wir bei \glqq Bälle und Kisten\grqq\ $L_i$ (Anzahl
der Bälle in Kiste $i$) und $X_i^j$ (Ball $i$ landet in Kiste $j$). Interesse
besteht, wie scharf man (m.h.W.) vom Erwartungswert abweicht.
\begin{satz}[Markov-Ungleichung]
	Sei $X$ eine nichtnegative Zufallsvariable. Dann gilt für alle $t > 0$:
	\[
	  \Pr[X \geq t] \leq \frac{\E[X]}{t}.
	\]
	Mit $t = c \cdot \E[X]$ für $c \geq 1$ ergibt sich
	\[
	  \Pr\left[X \geq c \cdot \E[X]\right] \leq \frac{1}{c}.
	\]
\end{satz}
\begin{proof}
	Sei $t > 0$ gegeben. Sei weiterhin 
	\[
	  \tau(X) = \begin{cases} 1 & \text{falls } \frac{X}{t} \geq 1 \\ 0 &
		  \text{sonst} \end{cases}
	\]
	Dann ist
	\[
	  \E[\tau(X)] = \Pr[\tau(X)=1] = \Pr\left[\frac{X}{t} \geq 1\right] = \Pr[X
	  \geq t] \leq E\left[\frac{X}{t}\right] = \frac{1}{t} \E[X].
	\]
\end{proof}
\begin{defn}
	Sei $X : \Omega \mapsto \mathbb{R}$ eine Zufallsvariable. Die \emph{Varianz} von $X$ ist
	\[
	  \Var[X] = E\left[\left(X - \E[X]\right)^2 \right] = E\left[X^2\right] - \E[X]^2
	\]
	Die \emph{Standardabweichung} $\sigma$ ist die Quadratwurzel der \emph{Varianz}:
	\[
	  \sigma[X] = \sqrt{\Var[X]}
	\]
\end{defn}
\begin{satz}[Rechenregeln für die Varianz]
	Seien $X$ und $Y$ unabhängige Zufallsvariablen und $c$ eine beliebige Konstante.
	\begin{enumerate}[(a)]
		\item $\Var[X+Y] = \Var[x] + \Var[Y]$
		\item $\Var[c\cdot X] = c^2 \Var[X]$
		\item Ist $X$ eine $0$-$1$-Zufallsvariable, so ist $\Var[X] =
			\E[X] \cdot \left(1 - \E[X]\right)$
	\end{enumerate}
\end{satz}
\begin{satz}[Chebychev]
	Sei $X$ eine Zufallsvariable. Für jedes $\varepsilon > 0$ gilt:
	\[
	  \Pr\left[\,|X-\E[X]| \geq \varepsilon\,\right] \leq \frac{\Var[X]}{\varepsilon^2}
	\]
	\begin{equation*}
		\begin{array}{ll}
		\varepsilon := k\,\sqrt{\Var[X]} = k\,\sigma[X] &\Rightarrow
		\Pr\left[\,|X - \E[X]| \geq k\,\sigma[X]\,\right] \leq
		\frac{1}{k^2} \\
		\varepsilon := \varepsilon^\prime\,\E[X] &\Rightarrow \Pr\left[\,|X-\E[X]| \geq
		\varepsilon^\prime \E[X]\,\right] \leq
		\frac{1}{\left(\varepsilon^\prime\right)^2}\cdot\frac{\Var[X]}{\E[X]^2}
		\end{array}
	\end{equation*}
\end{satz}
\begin{proof}
	\begin{align*}
	  \Var[X] &= \sum_{\omega \in \Omega} \left(X(\omega) - \E[X]\right)^2
	  \cdot \Pr[\omega] \\ 
	  &\geq
	  \sum_{\substack{\omega\in\Omega\\\left(X(\omega)-\E[X]\right)^2\geq\varepsilon^2}}
	  \left(X(\omega) - \E[X]\right)^2 \cdot \Pr[\omega]\\
	  &\geq \sum_{\substack{\omega\in\Omega\\|X(\omega)-\E[X]|\geq\varepsilon}}
	  \varepsilon^2 \cdot \Pr[\omega] = \varepsilon^2 \Pr\left[\,|X-\E[X]| \geq
	  \varepsilon\,\right]
	\end{align*}
\end{proof}
\begin{satz}[Chernoff-Schranken]
	Seien $X_1, \dots, X_n$ unabhängige $0$-$1$-Zufallsvariablen. Dann gilt
	für $X = \sum_{i=1}^n X_i$ und $\E[X] = \sum_{i=1}^n \Pr[X_i = 1]$ und
	jedes $\varepsilon, \ 0 < \varepsilon \leq 1$:
	\begin{enumerate}[(a)]
		\item $\Pr\left[\,X < (1-\varepsilon)\,\E[X]\,\right] <
			e^{-\E[X]\cdot\frac{\varepsilon^2}{2}}$
		\item $\Pr\left[\,X > (1+\varepsilon)\,\E[X]\,\right] <
			e^{-\E[X]\cdot\frac{\varepsilon^2}{4}}$
		\item $\Pr\left[\,|X-\E[X]| \leq \varepsilon\,\E[X]\,\right] \geq 1 -
			2\,e^{-\E[X]\cdot\frac{\varepsilon^2}{4}}$
	\end{enumerate}
\end{satz}
\begin{proof}
	\begin{enumerate}[(a)]
		\item Für jedes $t > 0$ gilt:
			\begin{align*}
			  P&\left[\,X < (1-\varepsilon)\,\E[X]\,\right] =
			  \Pr\left[\,-X > -(1-\varepsilon)\,\E[X]\,\right] =
			  \Pr\left[e^{-t\,X}>e^{-t\,(1-\varepsilon)\,\E[X]}\,\right]\\
			  \intertext{Mit der Markov-Ungleichung und der
			  Unabhängigkeit der $X_i$ ergibt sich weiter}
			  &\leq
			  \frac{E\left[e^{-t\,X}\right]}{e^{-t\,(1-\varepsilon)\,\E[X]}}
			  = e^{t\,(1-\varepsilon)\,\E[X]}\cdot
			  E\left[\,\prod_{j=1}^{n} e^{-t\,X_j}\,\right] =
			  e^{t\,(1-\varepsilon)\,\E[X]}\,\prod_{j=1}^{n}
			  E\left[e^{-t\,X_j}\right]\\
			  &= e^{t\,(1-\varepsilon)\,\E[X]}\,\prod_{i=1}^{n}
			  \left(\Pr[X_i=1]\cdot e^{-t} +
			  \left(1-\Pr[X_i=1]\right)\, e^0\right) \\
			  &= e^{t\,(1-\varepsilon)\,\E[X]}\,\prod_{i=1}^{n}
			  \left(1 + \Pr[X_i=1]\cdot(e^{-t} - 1)\right) \\
			  \intertext{Wähle $t=-\ln(1-\varepsilon)$ und benutze,
			  dass $1-x\leq e^{-x}$}
			  &\leq (1-\varepsilon)^{-(1-\varepsilon)\,\E[X]}\,
			  \prod_{i=1}^{n}\left(1-\varepsilon\,\Pr[X_i=1]\right) \\
			  &\leq (1-\varepsilon)^{-(1-\varepsilon)\,\E[X]}\,
			  \prod_{i=1}^{n} e^{-\varepsilon\,\Pr[X_i=1]} \\
			  &= (1-\varepsilon)^{-(1-\varepsilon)\,\E[X]}\,
			  e^{\sum_{i=1}^{n} -\varepsilon\,\Pr[X_i=1]} \\
			  \intertext{für $0<\varepsilon\leq 1$ ist
			  $(1-\varepsilon)^{-(1-\varepsilon)} < e^{\varepsilon
			  - \frac{\varepsilon^2}{2}}$ (Anfang der
			  McLaurin-Reihe)}
			  &< e^{-\E[X]\,\frac{\varepsilon^2}{2}}
			\end{align*}
		\item analog zu (a)
		\item Folgerung aus (a) und (b)
	\end{enumerate}
\end{proof}

\section{Markov-Ketten und random walks}
\subsection{Randomisierte Algorithmen für $2$- und $3$-SAT}
\subsection{\emph{Random walks} auf ungerichteten Graphen und Erreichbarkeit}
\ohead{16. Juni 2011}
\begin{defn}
	Sei $G$ ein ungerichteter Graph. Ein \emph{random walk} auf $G$ ist
	eine (durch eine Markov-Kette beschriebene) zufällige Reise eines
	Partikels auf den Knoten von $G$. Ist ein Partikel zu einem Zeitpunkt
	im Knoten $i$ mit Grad $d(i)$, dann wechselt das Partikel mit
	Wahrscheinlichkeit jeweils $\frac{1}{d(i)}$ zu einem der $d(i)$
	Nachbarknoten von $i$.
\end{defn}
\paragraph{Beobachtung:} Ein \emph{random walk} ist aperiodisch genau dann,
wenn $G$ nicht bipartit ist.

\begin{satz}
	Ein \emph{random walk} auf $G$ (zusammenhängend, aperiodisch)
	konvergiert gegen die stationäre Verteilung $\pi$ mit $\pi_i =
	\frac{d(i)}{2\,|E|}$ für alle Knoten $i$.
\end{satz}
\begin{proof}
	\[ \sum \pi_i = \sum \frac{d(i)}{2\,|E|} =
	\frac{1}{2\,|E|}\,\underbrace{\sum d(i)}_{=2\,|E|} = 1 \]
	% Sei $M$ die zugehörige Markov-Kette. % taucht nie wieder auf …
	Sei $\Gamma(i)$ die Menge der Nachbarknoten von $i$.
	\[ \pi_i = \sum_{j\in\Gamma(i)} \frac{d(j)}{2\,|E|}\,\frac{1}{d(j)} =
	\frac{1}{2\,|E|}\,\sum_{j\in\Gamma(i)} 1 = \frac{d(i)}{2\,|E|} \]
\end{proof}
\begin{korr}
	Sei $h_{i,i}$ die erwartete Zeit, die vergeht, um von $i$ wieder nach
	$i$ zurückzukommen. Dann ist
	\[
	  h_{i,i} = \frac{2\,|E|}{d(i)}
	\]
\end{korr}
\begin{lemm}
	Entsprechend ist $h_{i,j}$ die erwartete Anzahl von Schritten, um von
	$i$ nach $j$ zu kommen, für $\{i,j\} \in E$, und es gilt
	\[
	  h_{i,j} < 2\,|E|
	\]
\end{lemm}
\begin{proof}
	\[
	  \frac{2\,|E|}{d(j)} = h_{j,j} =
	  \frac{1}{d(j)}\,\sum_{k\in\Gamma(j)}\left(1+h_{k,j}\right)
	  \ \ \Rightarrow\ \ 2\,|E| = \sum_{k\in\Gamma(j)} \left(1+ h_{k,j}\right) >
	  h_{i,j} \qedhere
	\]
\end{proof}
\begin{defn}
	Die \emph{CoverTime} eines Graphen ist das Maximum über alle Knoten $i$
	der erwarteten Zeit, um alle Knoten von $G$, gestartet bei $i$, zu
	besuchen.
\end{defn}
\begin{satz}
	Die \emph{CoverTime} von $G$ ist kleiner als $4\cdot|V|\cdot|E|$.
\end{satz}
\begin{proof}
	Wähle einen beliebigen Spannbaum von $G$. In seiner Eigenschaft als
	Spannbaum hat dieser $|V|-1$ Kanten und damit dauert das Traversieren
	maximal $2\cdot\left(|V|-1\right)$ Schritte. Das Fortschreiten von
	einem Knoten $i$ zu einem Nachbarn $j$ dauert wie oben gezeigt weniger
	als $2\,|E|$ Schritte; es ergibt sich folglich eine obere Grenze von
	$2\cdot\left(|V|-1\right)\cdot2\,|E| = 4\,|V|\,|E|$.
\end{proof}
\paragraph{Anwendung} $s$-$t$-Connectivity. Gegeben ein Graph $G = (V,E)$ und
zwei Knoten $s, t \in V$, gibt es einen Weg von $s$ nach $t$?

Mittels regulärer Tiefensuche lässt sich diese Frage in $\Theta(n)$ Platz und
$\mathcal{O}(n^2)$ Zeit beantworten.
\begin{algorithm}[H]
	\caption{$s$-$t$-Connectivity}
	\vspace{0.4cm}
	\begin{itemize}
		\setlength{\itemsep}{2pt}
		\setlength{\parskip}{2pt}
		\setlength{\parsep}{2pt}
		\item Starte einen \emph{random walk} bei $s$
		\item falls er in $4n^3$ Schritten $t$ erreicht hat, gib \glqq
			ja\grqq\ aus, sonst \glqq nein\grqq
	\end{itemize}
\end{algorithm}
\begin{satz}
	$s$-$t$-\emph{Connectivity} gibt mit Wahrscheinlichkeit $\frac{1}{2}$
	die korrekte Antwort und kann sich nur im \glqq nein\grqq-Fall irren.
\end{satz}
\begin{proof}
	Die Anzahl der Kanten in einem einfachen Graphen ist maximal
	$\binom{n}{2} = \frac{n(n-1)}{2}$. Die \emph{CoverTime} ist somit
	maximal $2n^3$. Der Algorithmus führt maximal $4n^3$ Schritte aus und
	es folgt mit der Markov-Ungleichung, dass die Wahrscheinlichkeit einer
	fehlerhaften Ausgabe von \glqq nein\grqq\ kleiner oder gleich
	$\frac{1}{2}$ ist.
	\[
	  \Pr\left[\text{$s$-$t$-Connectivity gibt \glqq nein\grqq\ aus}\ |\
	  \text{$t$ von $s$ aus erreichbar}\right] \leq \frac{2n^3}{4n^3} =
	  \frac{1}{2}
	\]
\end{proof}

\section{Die Probabilistische Methode}
Besitzt kein Objekt aus einer Menge von Objekten eine gewisse Eigenschaft $E$,
so ist die Wahrscheinlichkeit, dass ein zufällig gewähltes Objekt die
Eigenschaft $E$ besitzt, gleich null. Im Umkehrschluss muss es Objekte mit der
Eigenschaft $E$ geben, falls die Wahrscheinlichkeit positiv ist (auch, falls
diese verschwindend klein ist), oder anders gesagt, falls für ein zufällig
gewähltes Objekt $A$ gilt, dass
\[
  \Pr\left[\,A \text{ besitzt die Eigenschaft } E \text{ nicht}\,\right] < 1.
\]
\subsection{Das \emph{max-cut}-Problem}
Gegeben sei ein zusammenhängender Graph $G=\left(V, E\right)$. Gesucht ist eine
Partition von $V$ in $A$ und $B$, sodass möglichst viele Kanten zwischen Knoten
aus $A$ und Knoten aus $B$ verlaufen. Die Anzahl $C\left(A,B\right)$ ist die
Größe des Schnitts $\left[A,B\right]$.

Die Entscheidungsvariante (\glqq Gibt es einen Schnitt der Größe $k$?\grqq) ist
$\mathcal{NP}$-vollständig\footnote{\emph{MinCut} ist in $\mathcal{P}$.}.

Im Folgenden sei $|V| = n$ und $|E| = m$.

\begin{satz}
	Es gibt immer einen Schnitt $\left[A,B\right]$ in $G$ mit $C\left( A,B
	\right) \geq \frac{m}{2}$.
\end{satz}
\begin{proof}
	Betrachte den folgenden Algorithmus:
	\begin{algorithm}[H]
		\caption{RandomCut (\emph{MonteCarlo})}
		\vspace{0.4cm}
		\begin{enumerate}
			\setlength{\itemsep}{2pt}
			\setlength{\parskip}{2pt}
			\setlength{\parsep}{2pt}
			\item \textbf{for $i=1$ to $n$ do}
			\item[] $\hphantom{for}$ mit Wahrscheinlichkeit
				$\frac{1}{2}$: lege $v_i$ nach $A$
			\item[] $\hphantom{for}$ mit Wahrscheinlichkeit
				$\frac{1}{2}$: lege $v_i$ nach $B$
			\item[] \textbf{done}
			\item gib $\left[ A,B \right]$ aus
		\end{enumerate}
	\end{algorithm}

	Sei $e_1, e_2, \dots, e_m$ eine beliebige Aufzählung der Kanten aus
	$E$ und
	\[
	  X_j = \begin{cases} 1 & \text{falls $e_i$ über den Schnitt geht} \\ 0 &
		\text{sonst} \end{cases}
	\]
	Dann ist $C(A,B) = \sum_{j=1}^m X_j$ und
	\[
	  \Pr[X_j = 1] = \Pr\left[e_j = \{v, w\} \text{ geht über den
	  Schnitt}\right] = \frac{1}{2},
	\]
	da genau $2$ von $4$ Möglichkeiten, $v$ und $w$ auf $A$ und $B$
	aufzuteilen, zu einer Kante über den Schnitt führen. Für den
	Erwartungswert ergibt sich
	\[
	  \E\left[\,C(A,B)\,\right] = \sum_{j=1}^m E[X_j] =
	  \sum_{j=1}^m \Pr[X_j = 1] = \sum_{j=1}^m \frac{1}{2} = \frac{m}{2}.
	\]
	Da $X$ offensichtlich Werte kleiner als $\frac{m}{2}$ annehmen kann
	muss es also auch Fälle geben, in denen $X$ Werte größer als
	$\frac{m}{2}$ annimmt.
\end{proof}
\begin{algorithm}[H]
	\caption{LargeCut (\emph{Las-Vegas})}\label{alg:largecut}
	\vspace{0.4cm}
	\begin{enumerate}
			\setlength{\itemsep}{2pt}
			\setlength{\parskip}{2pt}
			\setlength{\parsep}{2pt}
		\item $t := 0$
		\item[] \textbf{repeat}
		\item $\hphantom{repeat}$ $t:=t+1$
		\item $\hphantom{repeat}$ würfle einen Schnitt $[A,B]$
		\item[] \textbf{until} $C(A,B) \geq \frac{m}{2}$
	\end{enumerate}
\end{algorithm}
Gesucht ist die erwartete Laufzeit $\E[t]$.
\[
p = \Pr\left[\,C(A,B) \geq \frac{m}{2}\,\right]
\]
Mit dem Ergebnis des vorigen Beweises ergibt sich
\begin{align*}
  \frac{m}{2} &= \E\left[\,C(A,B)\,\right] = \sum_{i=0}^{\infty} i \cdot
  \Pr\left[C(A,B)=i\right] \\
  &= \sum_{i \leq \frac{m}{2} - 1} i \cdot \Pr\left[ C(A,B) = i \right] +
  \sum_{i \geq \frac{m}{2}} i \cdot \Pr\left[ C(A,B) = i \right] \\
  &\leq \left( 1-p \right)\left( \frac{m}{2}-1 \right) + p\cdot m
\end{align*}
und folglich
\[
  p \geq \frac{1}{\frac{m}{2} + 1}.
\]
Der Erwartungswert ist also (geometrische Verteilung)
\[
  \E[t] \leq \frac{m}{2} + 1
\]
und  die gesamte Laufzeit von \autoref{alg:largecut} ist
\[
  \mathcal{O}\left( \left( \frac{m}{2} + 1 \right)\cdot(n+m) \right) =
  \mathcal{O}(m^2)
\]

\subsection{\emph{Independent sets} und \emph{Sample \& Modify}}
Gegeben sei ein Graph $G = \left( V,E \right)$. Eine Teilmenge $U$ von $V$
heißt unabhängig, falls es keine Kante in $E$ gibt, die Knoten aus $U$
verbindet.

Gesucht ist eine möglichst große Menge $U$. Die Entscheidungsvariante ist
$\mathcal{NP}$-vollständig. Wie zuvor sei $|V|=n$, $|E|=m$.

\begin{satz}
	$G$ hat eine unabhängige Menge $U$ mit $|U| \geq \frac{n^2}{4m}$.
\end{satz}
\begin{proof}
	Sei $d$ der durchschnittliche Knotengrad: $\displaystyle d= \frac{2m}{n}$
	\begin{algorithm}[H]
		\caption{}
		\vspace{0.4cm}
		\begin{enumerate}
			\setlength{\itemsep}{2pt}
			\setlength{\parskip}{2pt}
			\setlength{\parsep}{2pt}
		\item \textbf{for $i=1$ to $n$ do} \hfill \glqq\emph{sample}\grqq
		\item[] $\hphantom{for}$ lösche $v_i$ und seine Kanten aus $G$
			mit Wahrscheinlichkeit $1-\frac{1}{d}$
		\item Für jede übrig gebliebene Kante \hfill \glqq\emph{modify}\grqq
		\item[] $\hphantom{for}$ lösche sie und einen ihrer Knoten
		\end{enumerate}
	\end{algorithm}
	Die übrig gebliebenen Knoten bilden eine unabhängige Menge. Wie viele
	Knoten bleiben in Schritt $1$ übrig? Sei $X$ deren Zahl: $\E[X] =
	\frac{n}{d}$.

	Sei $Y$ die Anzahl der Kanten, die Schritt $1$ überleben. Eine Kante
	überlebt genau dann, wenn ihre beiden Knoten überleben:
	\[
	  \E[Y] = \frac{1}{d}\frac{1}{d}\cdot m =
	  \frac{1}{d^2}\cdot\frac{n\,d}{2} = \frac{n}{2d}
	\]
	Der erwartete Wert für die Größe der Ausgabe ist damit
	\[
	  E[X-Y] = \frac{n}{d} - \frac{n}{2d} = \frac{n^2}{4m}.
	\]
\end{proof}

\ohead{7. Juli 2011}
\section{Counting und die Monte-Carlo-Methode}
Wichtig: Stichproben ziehen, Sampling
\subsection{Kombinatorische Zählprobleme und $\#P$-Vollständigkeit}
\begin{defn}
	Bei einem kombinatorischen Zählproblem $\#\Pi$ ist ein "`normales"'
	kombinatorisches Optimierungsproblem $\Pi$ gegeben. Die Aufgabe besteht
	darin, die Anzahl $\#(I)$ der zur Eingabe $I$ gehörenden zulässigen
	Lösungen zu bestimmen.
\end{defn}
\begin{exmp}
	$\text{COL}_k$ ist das kombinatorische Optimierungsproblem, eine
	Färbung des Eingabe-Graphen mit möglichst wenigen Farben zu bestimmen,
	wobei höchstens $k$ Farben zulässig sind.

	$\#\text{COL}_k$ ist die Anzahl der zulässigen Färbungen (nicht der optimalen).
\end{exmp}
\begin{defn}
	\begin{enumerate}[(a)]
		\item	$\#\text{DNF}$: Probleminstanz $\Psi$ ist eine
			Boolessche Formel in Disjunkter Normalform über den
			Variablen $V = \{x_1, \dots, x_n\}$, d.h. $\Psi = C_1
			\vee C_2 \vee \dots \vee C_m$, und die $C_i$ sind
			Monome, d.h. "`Ver-UND-ungen"' von Literalen. Eine
			zulässige Belegung ist eine Belegung der Variablen, die
			$\Psi$ erfüllt. Gesucht ist die Anzahl der erfüllenden
			Belegungen.
		\item	$\#\text{COL}_k$ (exakte Def. später)
	\end{enumerate}
\end{defn}
\begin{defn}
	Die Komplexitätsklasse $\#\text{P}$ ist die Menge der kombinatorischen
	Zählprobleme, für die es je Eingabe höchstens polynomiell viele
	akzeptierende Rechnungen einer nicht-de\-ter\-mi\-nis\-tischen
	Turingmaschine gibt.
\end{defn}
$\#\text{DNF}$ ist $\#\text{P}$-vollständig.
\subsection{Relative Gütegarantie, die Expansion und die Wahl des Universums}
Ein Approximationsalgorithmus $\mathcal{A}$ für $\#\Pi$ hat bei Eingabe $I$ die
\textit{individuelle relative Güte}
\[ \rho_\mathcal{A}(I) = \max\left\{ \frac{\mathcal{A}(I)}{\#(I)},
\frac{\#(I)}{\mathcal{A}(I)} \right\}. \]
\begin{defn}
	Sei $I$ eine Eingabe von $\#\Pi$. Die Menge der zulässigen Lösungen sei
	$S(I)$ (d.h. gesucht ist $\#(I) = |S(I)|$). Sei $U_I$ eine Menge mit
	$S(I) \subseteq U_I$. $U_I$ wird \textit{Universum} von $S(I)$ genannt.
	Sei $\xi = \frac{|U_I|}{|S(I)|}$ das Verhältnis der beiden
	Kardinalitäten. $\xi$ ist die \textit{Expansion} des Universums.
\end{defn}
\begin{satz}
	Sei $s$ eine Zahl mit $\xi \leq s$. Dann approximiert $\mathcal{A}(I) =
	\frac{|U_I|}{\sqrt{s}}$ den gesuchten Wert $\#(I)$ mit relativer Güte
	$\sqrt{s}$.
\end{satz}
$\#\text{DNF}$: Zwei wichtige Eigenschaften
\begin{enumerate}[(1)]
	\item Wir können erfüllende Belegungen ganz einfach finden, da es
		ausreicht, ein Monom zu erfüllen.
	\item Wir können für ein Monom $C$ die Zahl $\#(C)$ unmittelbar
		berechnen.
\end{enumerate}
\begin{lemm}
	Sei $C = l_1 \wedge \dots \wedge l_k$ ein Monom der DNF $\Psi$ über $n$
	Variablen.
	Dann gibt es genau $2^{n-k}$ Belegungen, die $C$ erfüllen, d.h. $\#(C) =
	2^{n-k}$.
\end{lemm}
Sei $k^*$ die Länge des kürzesten Monoms in $\Psi$: $\#(\Psi) \geq 2^{n-k^*}$
\[ U_\Psi^{\text{blind}} = \left\{ u\,|\,u \text{ ist eine Belegung der
Variablen} \right\}; \quad \left|U_\Psi^{\text{blind}}\right| = 2^n \]
\[ \xi_{\text{blind}} = \frac{\left|U_\Psi^{\text{blind}}\right|}{\#(\Psi)} \leq \frac{2^n}{2^{n-k^*}} = 2^{k^*} \]
\[ \mathcal{A}_1(\Psi) = \frac{2^n}{\sqrt{2^{k^*}}} = 2^{n-k^*/2} \]
\[ \Psi_{\text{bad}} = x_1 \wedge x_2 \wedge \dots \wedge x_{n/2} \quad \quad
k^* = \frac{n}{2} \quad \quad \#(\Psi_{\text{bad}}) = 2^{n/2} \quad \quad
\xi_{\text{blind}} = 2^{n/2} \]
\[ \mathcal{A}_1(\Psi_{\text{bad}}) = 2^{\frac{3}{4}n}, \quad \text{Abweichung: }\,2^{n/4} \]
\begin{eqnarray*}
	S(\Psi) = & \bigcup_{j=1}^{m} \left\{ u\,|\,u \text{ erfüllt } C_j
	\right\} & = \bigcup_{j=1}^{m} \left\{ u\,|\,u \text{ erfüllt } C_j,
	\text{ aber kein } C_k \text{ mit } k < j \right\} \\
	S'(\Psi) = & \dots & = \bigcup_{j=1}^{m} \left\{ (u, j)\,|\,u \text{ erfüllt } C_j,
	\text{ aber kein } C_k \text{ mit } k < j \right\}
\end{eqnarray*}
\[ \#(\Psi) = |S(\Psi)| = |S'(\Psi)| \]
\[ U_{\Psi}^{\text{clever}} = \left\{ u\,|\,u \text{ macht } C_j \text{ wahr} \right\} =
\bigcupdot_{j=1}^{m} \left\{ (u, j)\,|\,u \text{ macht } C_j \text{ wahr} \right\} \]
\end{document}
