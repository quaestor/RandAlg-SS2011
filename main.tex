\documentclass[a4paper]{scrartcl}
\pagestyle{plain}
\usepackage{a4wide}
\usepackage[ngerman]{babel}
\usepackage{tikz}
\usepackage{enumerate}
\usepackage{listings}
\usepackage[colorlinks=true,linkcolor=black]{hyperref}
\usetikzlibrary{calc}
\usetikzlibrary{matrix}
\usetikzlibrary{automata}
\usetikzlibrary{decorations.pathreplacing}
\usetikzlibrary{positioning}
\usepackage{dsfont}
\usepackage{amsmath, amsthm, amssymb}
\usepackage{algorithm}
\usepackage{algpseudocode}
\usepackage{color}
\usepackage{scrpage2}
\usepackage[T1]{fontenc}
\usepackage[utf8]{inputenc}
%\usepackage[math,light]{anttor}
\usepackage{textcomp}
\usepackage{etoolbox}

% Pretty captions!
%\usepackage{caption}
%\DeclareCaptionFont{white}{\color{white}}
%\DeclareCaptionFormat{listing}{\colorbox{gray}{\parbox{0.98\linewidth}{#1#2#3}}}
%\captionsetup[lstlisting]{format=listing,labelfont=white,textfont=white}

% Pretty page headings!
\pagestyle{scrheadings}
\clearscrheadings
\clearscrplain
\setfootwidth{head}
\ofoot{\pagemark}

\makeatletter
% Different format for headings
\def\section{\@startsection{section}{1}%
  \z@{.7\baselineskip\@plus\baselineskip}{.5\baselineskip}%
  {\normalfont\Large\scshape\centering}}
% This does spacing around caption.
\setlength{\abovecaptionskip}{0.5em}
\setlength{\belowcaptionskip}{0.5em}
% This does justification (left) of caption.
\long\def\@makecaption#1#2{%
  \vskip\abovecaptionskip
  \sbox\@tempboxa{#2}%
  \ifdim \wd\@tempboxa >\hsize
    #2\par
  \else
    \global \@minipagefalse
    \hb@xt@\hsize{\box\@tempboxa\hfill}%
  \fi
  \vskip\belowcaptionskip}
\providerobustcmd*{\bigcupdot}{%
  \mathop{%
    \mathpalette\bigop@dot\bigcup
  }%
}
\newrobustcmd*{\bigop@dot}[2]{%
  \setbox0=\hbox{$\m@th#1#2$}%
  \vbox{%
    \lineskiplimit=\maxdimen
    \lineskip=-0.7\dimexpr\ht0+\dp0\relax
    \ialign{%
      \hfil##\hfil\cr
      $\m@th\cdot$\cr
      \box0\cr
    }%
  }%
}
\makeatother

\setkomafont{title}{\normalfont}
\setkomafont{pageheadfoot}{\normalfont}
\setkomafont{pagenumber}{\normalfont\scshape}
\setkomafont{disposition}{\normalfont\bfseries}

\widowpenalty=300
\clubpenalty=300

\newtheorem*{defn}{Definition}
\newtheorem*{exmp}{Beispiel}
\newtheorem*{satz}{Satz}
\newtheorem*{lemm}{Lemma}

\renewcommand{\algorithmicrequire}{\textbf{Input:}}
\renewcommand{\algorithmicensure}{\textbf{Output:}}

\title{Randomisierte Algorithmen\\
       Sommersemester 2011 \\
       \small Mitschrift der Tafelanschrift von Prof. Wanka}

\begin{document}
\ohead{5. Mai 2011}
\section{Beispiele (\glqq zum Warmwerden\grqq)}
\subsection{Randomized Quicksort}
\begin{algorithm}
	\caption{randQS ($S$: Array aus $n$ verschiedenen Zahlen)}
	\vspace{0.4cm}
	\begin{enumerate}
		\setlength{\itemsep}{2pt}
		\setlength{\parskip}{2pt}
		\setlength{\parsep}{2pt}
		\item Wähle ein $y$ aus $S$ ($y$ heißt Pivotelement) u.a.r.\ (\textit{uniformly at random})
		\item $S_1 := \{ x \in S\ |\ x < y \} \quad S_2 := \{ x \in S\ |\ x > y \}$
		\item \textbf{if} $S_1 \not= \emptyset$ \textbf{then} randQS($S_1$)
		\item print $y$
		\item \textbf{if} $S_2 \not= \emptyset$ \textbf{then} randQS($S_2$)
	\end{enumerate}
\end{algorithm}
Wir messen die Laufzeit in der Anzahl der Vergleiche auf Schlüsseln.
\subsubsection{\textit{worst-case}-Szenario}
\begin{center}
\renewcommand\arraystretch{1.3}
\begin{tabular}{ccccccc|c|c}
	&&&&&&& \# Vergleiche & Wahrscheinlichkeit \\
	$1$&$2$&$3$&$4$&$5$&$6$&$7$&$6$&$\frac{1}{7}$\\
	&$2$&$3$&$4$&$5$&$6$&$7$&$5$&$\frac{1}{6}$\\
	&&&$\ddots$&&&&&\\
	&&&&&$6$&$7$&$1$&$\frac{1}{2}$\\
	&&&&&&$7$&$0$&$1$
\end{tabular}
\renewcommand\arraystretch{2.5}
\begin{tabular}{rl}
	Gesamtzahl der Vergleiche: & $\displaystyle \sum_{i=1}^{n-1} i = \frac{n(n-1)}{2} \in \Theta(n^2)$ \\
	Gesamtwahrscheinlichkeit: & $\displaystyle \prod_{i=1}^{n} \frac{1}{i} = \frac{1}{n!}$
\end{tabular}
\end{center}
\subsubsection{Probabilistische Laufzeitanalyse}
Wir bestimmen die erwartete Laufzeit: Sei $s_i$ der Schlüssel mit Rang $i$
($i$-t-kleinster Schlüssel). Wir definieren für $1 \leq i < j \leq n$ folgende
Zufallsvariable:
\[
  X_{ij} = \begin{cases} 1 & \text{falls der Algor. $s_i$ und $s_j$ vergleicht} \\
	  0 & \text{sonst} \end{cases}
\]
Die Gesamtzahl der Vergleiche ist $\sum_{i=1}^n \sum_{j=i+1}^n X_{ij}$.
Definiere $p_{ij} = \Pr\left[X_{ij} = 1\right]$. Der Erwartungswert ist damit
wegen $E\left[X_{ij}\right] = 0 \cdot \Pr\left[X_{ij} = 0\right] + 1 \cdot
\Pr\left[X_{ij} = 1\right]$ gleich
\[
  E\left[\sum_{i=1}^n \sum_{j=i+1}^n X_{ij}\right] = \sum_{i=1}^n
  \sum_{j=i+1}^n E\left[X_{ij}\right] = \sum_{i<j} p_{ij}.
\]

Anhand folgenden Beispiels soll eine äquivalente Formulierung für $X_{ij}$
gezeigt werden (fettgedruckte Zahlen sind die in diesem Schritt neu gewählten
Pivotelemente, unterstrichene Zahlen entsprechen den Pivotelementen der
vorhergehenden Schritte):

\vspace{0.5cm}
\begin{minipage}{0.6\linewidth}
\renewcommand\arraystretch{1.3}
\centering
\begin{tabular}{ccccccc}
	$\mathbf{3}$ & $5$ & $2$ & $1$ & $5$ & $7$ & $6$ \\
	$2$ & $\mathbf{1}$ & $\underline{3}$ & $4$ & $\mathbf{5}$ & $7$ & $6$ \\
	$\underline{1}$ & $\mathbf{2}$ & $\underline{3}$ & $\mathbf{4}$ & $\underline{5}$ & $7$ & $\mathbf{6}$ \\
	$\underline{1}$ & $\underline{2}$ & $\underline{3}$ & $\underline{4}$ & $\underline{5}$ & $\underline{6}$ & $\mathbf{7}$
\end{tabular}
\end{minipage}\hfill\begin{minipage}{0.3\linewidth}
\begin{tikzpicture}[scale=0.7,level distance=10mm,level/.style={sibling distance=20mm/#1}]
	\tikzstyle{every node}=[draw,scale=0.7]
	\node (a){$3$}
	child {node (b) {$1$} child { node (c) {$2$} }}
	child {node (d) {$5$} child { node (e) {$4$} } child { node (f) {$6$} child { node (g) {$7$} } } };
\end{tikzpicture}
\end{minipage}
\vspace{0.5cm}

Die Schlüssel $s_i$ und $s_j$ werden also genau dann miteinander verglichen,
wenn nur $s_i$ oder $s_j$ als erstes aus $S_{ij} = \{ s_i, \dots, s_j \}$ als
Pivotelement gewählt werden:
\[
  X_{ij} = \begin{cases} 1 & \text{falls $s_j$-Knoten Nachfolger des
  $s_i$-Knoten oder umgekehrt} \\ 0 & \text{sonst} \end{cases}
\]
\begin{align*}
  \Rightarrow p_{ij} &= \Pr\left[s_i \text{ wird als erstes Pivotelement aus $S_{ij}$ gewählt}\right] \\
  &+ \Pr\left[s_j \text{ wird als erstes Pivotelement aus $S_{ij}$ gewählt}\right] = \frac{2}{j-i+1}
\end{align*}
Unter Verwendung der Eigenschaft der harmonischen Reihe $\ln(i+1) \leq H_i =
\sum_{k=1}^i \frac{1}{k} \leq 1 + \ln i$ ergibt sich für den Erwartungswert
\begin{align*}
	\sum_{i<j} p_{ij} &= \sum_{i=1}^n \sum_{j=1}^n \frac{2}{j-i+1} = 2 \cdot \sum_{i=2}^n \sum_{j=2}^i \frac{1}{j} \\
			  &\leq 2 \cdot \sum_{i=1}^n \left(H_i - 1\right) \leq 2 n \ln n = (2 \ln 2) n \log_2 n \approx 1.386\dots n \log_2 n
\end{align*}
\renewcommand\arraystretch{1.5}
\begin{tabular}{r|ccccccc}
	& & $j = 2$ & 3 & 4 & $\cdots$ & $n-1$ & $n$ \\
	\hline
	$i = 1$ & $0$ & $\frac{2}{2}$ & $\frac{2}{3}$ & $\frac{2}{4}$ & $\cdots$ & $\frac{2}{n-1}$ & $\frac{2}{n}$ \\
	$2$ && $0$ & $\frac{2}{2}$ & $\frac{2}{3}$ & $\cdots$ & $\cdots$ & $\frac{2}{n-1}$ \\
	$3$ &&& $0$ & $\frac{2}{2}$ & $\cdots$ & $\cdots$ & $\frac{2}{n-2}$
\end{tabular}

%%
\newpage

\ohead{12. Mai 2011}
\subsection{Ein \textit{Min-Cut}-Algorithmus}
\begin{defn}
	In einem Multigraphen kann es mehr als eine Kante zwischen zwei Knoten
	geben. Ein Schnitt in einem zusammenhängenden Multigraphen $G$ ist eine
	Kantenmenge, deren Entfernung aus $G$ den Multigraphen $G$ in zwei oder
	mehr Zusammenhangskomponenten zerlegt. Ein minimaler Schnitt
	(\emph{min cut}) ist ein Schnitt kleinstmöglicher
	Kardinalität\footnote{Das \emph{Max-Flow}-Problem ist zum
	\emph{Min-Cut}-Problem äquivalent}.
\end{defn}
\begin{exmp}
	Anwendung der Operation \emph{contract} auf einen einfachen Graphen.

	\vspace{0.5cm}
	\begin{tikzpicture}[scale=0.95]
		\node[circle,draw,scale=0.9] (1) at (0,2) {$1$};
		\fill[black!30] (0,0) circle (0.40cm);
		\node[circle,draw,scale=0.9] (2) at (0,0) {$2$};
		\fill[black!30] (2,2) circle (0.40cm);
		\node[circle,draw,scale=0.9] (3) at (2,2) {$3$};
		\node[circle,draw,scale=0.9] (4) at (2,0) {$4$};
		\node[circle,draw,scale=0.9] (5) at (4,1) {$5$};
		\draw[line width=5pt,-,black!30] (2) -- (3);
		\draw (1) -- (2);
		\draw (1) -- (3);
		\draw (1) -- (4);
		\draw (2) -- (3);
		\draw (2) -- (4);
		\draw (3) -- (4);
		\draw (3) -- (5);
		\draw (4) -- (5);
		\draw[->] (5,1) to node[above] {$\textnormal{contract}(2,3)$} (7,1);
		\node[circle,draw,scale=0.9] (1b) at (8,2) {$1$};
		\node[circle,draw,scale=0.9] (23) at (10,2) {$2,3$};
		\node[circle,draw,scale=0.9] (4b) at (10,0) {$4$};
		\node[circle,draw,scale=0.9] (5b) at (12,1) {$5$};
		\draw (1b) -- (23);
		\path[draw] (1b) edge [bend left] (23);
		\draw (1b) -- (4b);
		\draw (23) -- (4b);
		\path[draw] (23) edge [bend left] (4b);
		\draw (23) -- (5b);
		\draw (4b) -- (5b);
		\draw[->] (1.5,-2.5) to node[above] {$\textnormal{contract}\left(1,(2,3)\right)$} (3.5,-2.5);
		\node[circle,draw,scale=0.9] (123) at (5,-1.5) {$1,2,3$};
		\node[circle,draw,scale=0.9] (4c) at (5,-3.5) {$4$};
		\node[circle,draw,scale=0.9] (5c) at (7,-2.5) {$5$};
		\draw (123) -- (4c);
		\draw (123) -- (5c);
		\draw (4c) -- (5c);
		\path[draw] (123) edge [bend left] (4c);
		\path[draw] (123) edge [bend right] (4c);
		\draw[->] (8,-2.5) to node[above] {\textnormal{contract}$\left(4,(1,2,3)\right)$} (11,-2.5);
		\node[circle,draw,scale=0.9] (1234) at (12.5,-2.5) {$\begin{matrix}1,2\\[-0.3cm]3,4\end{matrix}$};
		\node[circle,draw,scale=0.9] (5d) at (14.5,-2.5) {$5$};
		\path[draw] (1234) edge [bend left] (5d);
		\path[draw] (1234) edge [bend right] (5d);
	\end{tikzpicture}
\end{exmp}
\begin{defn}[Operation \emph{contract} einer Kante $\{x, y\}$]
	Ersetze $x$ und $y$ durch einen neuen Knoten $z$, lösche alle Kanten zwischen $x$ und $y$ und ersetze alle übrigen Kanten $\{x, v\}$ bzw. $\{y, v\}$ durch eine neue Kante $\{z, v\}$.
\end{defn}
\begin{algorithm}
	\caption{contract ($G$: Multigraph)}
	\vspace{0.4cm}
	\begin{enumerate}
		\setlength{\itemsep}{2pt}
		\setlength{\parskip}{2pt}
		\setlength{\parsep}{2pt}
		\item $H := G$
		\item \textbf{while} $H$ enthält mehr als $2$ Knoten \textbf{do}
		\item $\hphantom{while}$ wähle u.a.r.\ eine Kante $\{x, y\}$ aus $H$
		\item $\hphantom{while}$ contract$(x,y)$ in $H$
		\item[] \textbf{done}
		\item sei $C$ die Menge der Kanten in $H$ (rückgerechnet auf $G$)
		\item gib $C$ aus
	\end{enumerate}
\end{algorithm}
\subsubsection{Analyse der Erfolgswahrscheinlichkeit}
Sei $e_i$ die in der $i$-ten Iteration kontrahierte Kante, $1 \leq i \leq n-2$,
und sei $H_i$ der zugehörige Graph. $H_0 = G, H_{n-2}$ ist der letzte Graph.
\pagebreak
	\paragraph{Beobachtung 1:} Die Menge $C$ ist ein Schnitt von $H_0 = G$, da
		sie ein Schnitt von $H_{n-2}$ ist.\\[-0.75cm]
	\paragraph{Beobachtung 2:} In jedem Graphen $H_i$ ist die Größe eines
	\emph{min cuts} nicht kleiner als in $G$.\vspace{0.5cm}

Sei $K$ ein beliebiger \emph{min cut} in $G$. $K$ überlebt die Folge der
Kontraktionen genau dann, wenn keine der Kanten aus $K$ zur Kontraktion (in
jeder Iteration) ausgewählt wurde.
\[
  \Pr\left[C \text{ ist } min\ cut\right] \geq \Pr\left[C = K\right] = \Pr\left[
  \bigwedge_{1\leq i\leq n-2} \left(e_i \not\in K\right)\right] =
  \prod_{i=1}^{n-2} \Pr\left[e_i \not\in K\ \Big|\ \bigwedge_{1\leq j < i} \left(e_j
  \not\in K\right)\right]
\]

\begin{lemm}
	Für $1 \leq i \leq n-2$ gilt:
	\[ \Pr\left[e_i \in K | \bigwedge_{1\leq j < i} \left(e_j \not\in K\right) \right] \leq \frac{2}{n-i+1} \]
\end{lemm}
\begin{proof} Sei $k = |K|$ sowie $n = |V|$.
	\begin{itemize}
		\item Jeder Knoten des Eingabegraphen hat mindestens Grad $k$
			(da $K$ \emph{min cut}):
			\[
			  \frac{1}{2} \cdot |K| \cdot |V| \leq |E|
			\]
			Für die Wahrscheinlichkeit, dass im ersten Schritt eine
			der Kanten aus $K$ gewählt wird, gilt also
			\[
			  \Pr\left[e_1 \in K\right] \leq
			  \frac{|K|}{\frac{1}{2}\cdot |K| \cdot |V|} =
			  \frac{2}{n}
			\]
		\item $H_1$ hat $n-1$ Knoten, analog gilt also
			\[
			  \Pr\left[e_2 \in K\ |\ e_1 \not\in K\right] \leq
			  \frac{|K|}{\frac{1}{2}\cdot |K| \cdot (n-1)} =
			  \frac{2}{n-1}
			\]
		\item Für jedes $i$ mit $1 \leq i \leq n-2$ gilt somit die Ungleichung
			\[
			  \Pr\left[e_i \in K\ \Big|\ \bigwedge_{j=1}^{i-1}
			  \left(e_j \not\in K\right)\right] \leq
			  \frac{|K|}{\frac{1}{2}\cdot |K| \cdot
			  \left(n-(i-1)\right)} = \frac{2}{n-i+1}
			\]
	\end{itemize}
\end{proof}

Damit lässt sich nun die Wahrscheinlichkeit abschätzen, dass $C$ (die Ausgabe
des \emph{contract}-Algorithmus) ein \emph{min cut} ist:
\[
  \Pr\left[ C \text{ ist } min\ cut\right] \geq \prod_{1\leq i \leq n-2}
  \left(1-\frac{2}{n-i+1}\right) = \prod_{i=1}^{n-2} \frac{n-i-1}{n-i+1} = \frac{2}{n(n-1)}
\]

\begin{satz}
	\emph{contract($G$)} gibt mit Wahrscheinlichkeit mindestens $\frac{2}{n(n-1)}$ einen \emph{min cut} aus.
\end{satz}

\subsubsection{Wahrscheinlichkeitsverstärkung (\emph{probability amplification})}
Wiederhole \emph{contract($G$)} $\frac{1}{2} \cdot c \cdot n(n-1)\cdot \ln n$
Mal (für beliebige Konstante $c$) und gib das beste $C$ aus.

Die Wahrscheinlichkeit, dass kein \emph{min cut} gefunden wurde, ist:
\[
  \left(1-\frac{2}{n(n-1)}\right)^{\frac{1}{2}c\,n(n-1)\,\ln n} =
  \left(\left(1+\frac{-2}{n(n-1)}\right)^{n(n-1)}\right)^{\frac{1}{2}c\,\ln n}
  \leq e^{-c\,\ln n} = \frac{1}{n^c}
\]

\subsection{Verifikation von Matrix-Produkten}
Gegeben: drei $n \times n$-Matrizen $A$, $B$, $C$ über $\{0, 1\}$ mit
Multiplikation und Addition modulo $2$.

Jemand behauptet: $A\cdot B = C$. Berechne $A\cdot B$ und teste auf Gleichheit.
Laufzeit\footnote{Coppersmith/Winograd: $\mathcal{O}(n^{2.376})$}
$\mathcal{O}(n^3)$.

\begin{algorithm}
	\caption{Algorithmus VerifyMatrixProduct}
	\vspace{0.4cm}
	\begin{enumerate}
		\setlength{\itemsep}{2pt}
		\setlength{\parskip}{2pt}
		\setlength{\parsep}{2pt}
		\item Wähle $v = \left(v_1, \dots, v_n\right)^T \in \{0, 1\}^n$ zufällig
		\item $u := B\cdot v$
		\item $w := A\cdot u$
		\item $x := C\cdot v$
		\item \textbf{if} $w \not= x$ \textbf{then return} \glqq sind ungleich\grqq
		\item[] \textbf{else return} \glqq sind gleich\grqq
	\end{enumerate}
\end{algorithm}

Der Algorithmus \emph{VerifyMatrixProduct} besitzt Laufzeit $\mathcal{O}(n^2)$,
was linearer Zeit entspricht, da die Eingabe bereits $\mathcal{O}(n^2)$ groß ist.

Wenn $A\cdot B = C$ ist, dann gilt
\[
  x = C \cdot v = \left( A \cdot B \right) v = A \left( B \cdot v \right) = A \cdot u = w
\]
Ist jedoch $A\cdot B \not= C$, so sei
\[
  D = A\cdot B - C
\]
und es gilt $D \not= 0$, d.h.\ es gibt mindestens ein $d_{i,j} \not= 0$. Sei $v
= \left( v_1, \dots, v_{j-1}, \ast, v{j+1}, \dots, v_n \right)^T \in \{0,
1\}^{n-1}$ beliebig, aber fest.
\[
  d_{i,1} \cdot v_1 + \dots + d_{i,j-1} \cdot v_{j-1} + d_{i,j} \cdot v_j +
  d_{i,j+1} v_{j+1} + \dots + d_{i,n} \cdot v_n = 0
\]
Dies ist eine Gleichung, die nach $v_j$ aufgelöst werden kann (da $d_{i,j}
\not= 0$) und damit genau eine Lösung hat. Der Algorithmus hat zwei
Möglichkeiten zur Wahl von $v_j$.
\[
  \Pr\left[ A\cdot B \cdot v = C \cdot v \right] = \Pr\left[ \left(A\cdot B -
  C\right) v = 0 \right] = \Pr\left[ D \cdot v = 0 \right] \leq \frac{1}{2}
\]

Dies bedeutet, dass eine unerwünschte Ausgabe von \glqq sind gleich\grqq\ mit
Wahrscheinlichkeit kleiner $\frac{1}{2}$ auftritt (bei Gleichheit wird immer
korrekt \glqq sind gleich\grqq\ ausgegeben). Nach $c$-maliger Ausführung des
Algorithmus gilt
\[
  \Pr\left[ \text{$c$-fache Wiederholung zeigt nicht $A\cdot B \not= C$}
  \right] \leq \left(\frac{1}{2}\right)^c
\]

\setcounter{section}{5}
\ohead{7. Juli 2011}
\section{Counting und die Monte-Carlo-Methode}
Wichtig: Stichproben ziehen, Sampling
\subsection{Kombinatorische Zählprobleme und $\#P$-Vollständigkeit}
\begin{defn}
	Bei einem kombinatorischen Zählproblem $\#\Pi$ ist ein "`normales"'
	kombinatorisches Optimierungsproblem $\Pi$ gegeben. Die Aufgabe besteht
	darin, die Anzahl $\#(I)$ der zur Eingabe $I$ gehörenden zulässigen
	Lösungen zu bestimmen.
\end{defn}
\begin{exmp}
	$\text{COL}_k$ ist das kombinatorische Optimierungsproblem, eine
	Färbung des Eingabe-Graphen mit möglichst wenigen Farben zu bestimmen,
	wobei höchstens $k$ Farben zulässig sind.

	$\#\text{COL}_k$ ist die Anzahl der zulässigen Färbungen (nicht der optimalen).
\end{exmp}
\begin{defn}
	\begin{enumerate}[(a)]
		\item	$\#\text{DNF}$: Probleminstanz $\Psi$ ist eine
			Boolessche Formel in Disjunkter Normalform über den
			Variablen $V = \{x_1, \dots, x_n\}$, d.h. $\Psi = C_1
			\vee C_2 \vee \dots \vee C_m$, und die $C_i$ sind
			Monome, d.h. "`Ver-UND-ungen"' von Literalen. Eine
			zulässige Belegung ist eine Belegung der Variablen, die
			$\Psi$ erfüllt. Gesucht ist die Anzahl der erfüllenden
			Belegungen.
		\item	$\#\text{COL}_k$ (exakte Def. später)
	\end{enumerate}
\end{defn}
\begin{defn}
	Die Komplexitätsklasse $\#\text{P}$ ist die Menge der kombinatorischen
	Zählprobleme, für die es je Eingabe höchstens polynomiell viele
	akzeptierende Rechnungen einer nicht-de\-ter\-mi\-nis\-tischen
	Turingmaschine gibt.
\end{defn}
$\#\text{DNF}$ ist $\#\text{P}$-vollständig.
\subsection{Relative Gütegarantie, die Expansion und die Wahl des Universums}
Ein Approximationsalgorithmus $\mathcal{A}$ für $\#\Pi$ hat bei Eingabe $I$ die
\textit{individuelle relative Güte}
\[ \rho_\mathcal{A}(I) = \max\left\{ \frac{\mathcal{A}(I)}{\#(I)},
\frac{\#(I)}{\mathcal{A}(I)} \right\}. \]
\begin{defn}
	Sei $I$ eine Eingabe von $\#\Pi$. Die Menge der zulässigen Lösungen sei
	$S(I)$ (d.h. gesucht ist $\#(I) = |S(I)|$). Sei $U_I$ eine Menge mit
	$S(I) \subseteq U_I$. $U_I$ wird \textit{Universum} von $S(I)$ genannt.
	Sei $\xi = \frac{|U_I|}{|S(I)|}$ das Verhältnis der beiden
	Kardinalitäten. $\xi$ ist die \textit{Expansion} des Universums.
\end{defn}
\begin{satz}
	Sei $s$ eine Zahl mit $\xi \leq s$. Dann approximiert $\mathcal{A}(I) =
	\frac{|U_I|}{\sqrt{s}}$ den gesuchten Wert $\#(I)$ mit relativer Güte
	$\sqrt{s}$.
\end{satz}
$\#\text{DNF}$: Zwei wichtige Eigenschaften
\begin{enumerate}[(1)]
	\item Wir können erfüllende Belegungen ganz einfach finden, da es
		ausreicht, ein Monom zu erfüllen.
	\item Wir können für ein Monom $C$ die Zahl $\#(C)$ unmittelbar
		berechnen.
\end{enumerate}
\begin{lemm}
	Sei $C = l_1 \wedge \dots \wedge l_k$ ein Monom der DNF $\Psi$ über $n$
	Variablen.
	Dann gibt es genau $2^{n-k}$ Belegungen, die $C$ erfüllen, d.h. $\#(C) =
	2^{n-k}$.
\end{lemm}
Sei $k^*$ die Länge des kürzesten Monoms in $\Psi$: $\#(\Psi) \geq 2^{n-k^*}$
\[ U_\Psi^{\text{blind}} = \left\{ u\,|\,u \text{ ist eine Belegung der
Variablen} \right\}; \quad \left|U_\Psi^{\text{blind}}\right| = 2^n \]
\[ \xi_{\text{blind}} = \frac{\left|U_\Psi^{\text{blind}}\right|}{\#(\Psi)} \leq \frac{2^n}{2^{n-k^*}} = 2^{k^*} \]
\[ \mathcal{A}_1(\Psi) = \frac{2^n}{\sqrt{2^{k^*}}} = 2^{n-k^*/2} \]
\[ \Psi_{\text{bad}} = x_1 \wedge x_2 \wedge \dots \wedge x_{n/2} \quad \quad
k^* = \frac{n}{2} \quad \quad \#(\Psi_{\text{bad}}) = 2^{n/2} \quad \quad
\xi_{\text{blind}} = 2^{n/2} \]
\[ \mathcal{A}_1(\Psi_{\text{bad}}) = 2^{\frac{3}{4}n}, \quad \text{Abweichung: }\,2^{n/4} \]
\begin{eqnarray*}
	S(\Psi) = & \bigcup_{j=1}^{m} \left\{ u\,|\,u \text{ erfüllt } C_j
	\right\} & = \bigcup_{j=1}^{m} \left\{ u\,|\,u \text{ erfüllt } C_j,
	\text{ aber kein } C_k \text{ mit } k < j \right\} \\
	S'(\Psi) = & \dots & = \bigcup_{j=1}^{m} \left\{ (u, j)\,|\,u \text{ erfüllt } C_j,
	\text{ aber kein } C_k \text{ mit } k < j \right\}
\end{eqnarray*}
\[ \#(\Psi) = |S(\Psi)| = |S'(\Psi)| \]
\[ U_{\Psi}^{\text{clever}} = \left\{ u\,|\,u \text{ macht } C_j \text{ wahr} \right\} =
\bigcupdot_{j=1}^{m} \left\{ (u, j)\,|\,u \text{ macht } C_j \text{ wahr} \right\} \]
\end{document}
