\ohead{7. Juli 2011}
\section{Counting und die Monte-Carlo-Methode}
Wichtig: Stichproben ziehen, Sampling
\subsection{Kombinatorische Zählprobleme und $\#P$-Vollständigkeit}
\begin{defn}
	Bei einem kombinatorischen Zählproblem $\#\Pi$ ist ein "`normales"'
	kombinatorisches Optimierungsproblem $\Pi$ gegeben. Die Aufgabe besteht
	darin, die Anzahl $\#(I)$ der zur Eingabe $I$ gehörenden zulässigen
	Lösungen zu bestimmen.
\end{defn}
\begin{exmp}
	$\text{COL}_k$ ist das kombinatorische Optimierungsproblem, eine
	Färbung des Eingabe-Graphen mit möglichst wenigen Farben zu bestimmen,
	wobei höchstens $k$ Farben zulässig sind.

	$\#\text{COL}_k$ ist die Anzahl der zulässigen Färbungen (nicht der optimalen).
\end{exmp}
\begin{defn}
	\begin{enumerate}[(a)]
		\item	$\#\text{DNF}$: Probleminstanz $\Psi$ ist eine
			Boolessche Formel in Disjunkter Normalform über den
			Variablen $V = \{x_1, \dots, x_n\}$, d.h. $\Psi = C_1
			\vee C_2 \vee \dots \vee C_m$, und die $C_i$ sind
			Monome, d.h. "`Ver-UND-ungen"' von Literalen. Eine
			zulässige Belegung ist eine Belegung der Variablen, die
			$\Psi$ erfüllt. Gesucht ist die Anzahl der erfüllenden
			Belegungen.
		\item	$\#\text{COL}_k$ (exakte Def. später)
	\end{enumerate}
\end{defn}
\begin{defn}
	Die Komplexitätsklasse $\#\text{P}$ ist die Menge der kombinatorischen
	Zählprobleme, für die es je Eingabe höchstens polynomiell viele
	akzeptierende Rechnungen einer nicht-de\-ter\-mi\-nis\-tischen
	Turingmaschine gibt.
\end{defn}
$\#\text{DNF}$ ist $\#\text{P}$-vollständig.
\subsection{Relative Gütegarantie, die Expansion und die Wahl des Universums}
Ein Approximationsalgorithmus $\mathcal{A}$ für $\#\Pi$ hat bei Eingabe $I$ die
\textit{individuelle relative Güte}
\[
  \rho_\mathcal{A}(I) = \max\left\{ \frac{\mathcal{A}(I)}{\#(I)},
  \frac{\#(I)}{\mathcal{A}(I)} \right\}.
\]
\begin{defn}
	Sei $I$ eine Eingabe von $\#\Pi$. Die Menge der zulässigen Lösungen sei
	$S(I)$ (d.h. gesucht ist $\#(I) = |S(I)|$). Sei $U_I$ eine Menge mit
	$S(I) \subseteq U_I$. $U_I$ wird \textit{Universum} von $S(I)$ genannt.
	Sei $\xi = \frac{|U_I|}{|S(I)|}$ das Verhältnis der beiden
	Kardinalitäten. $\xi$ ist die \textit{Expansion} des Universums.
\end{defn}
\begin{satz}
	Sei $s$ eine Zahl mit $\xi \leq s$. Dann approximiert $\mathcal{A}(I) =
	\frac{|U_I|}{\sqrt{s}}$ den gesuchten Wert $\#(I)$ mit relativer Güte
	$\sqrt{s}$.
\end{satz}
$\#\text{DNF}$: Zwei wichtige Eigenschaften
\begin{enumerate}[(1)]
	\item Wir können erfüllende Belegungen ganz einfach finden, da es
		ausreicht, ein Monom zu erfüllen.
	\item Wir können für ein Monom $C$ die Zahl $\#(C)$ unmittelbar
		berechnen.
\end{enumerate}
\begin{lemm}
	Sei $C = l_1 \wedge \dots \wedge l_k$ ein Monom der DNF $\Psi$ über $n$
	Variablen.
	Dann gibt es genau $2^{n-k}$ Belegungen, die $C$ erfüllen, d.h. $\#(C) =
	2^{n-k}$.
\end{lemm}
Sei $k^*$ die Länge des kürzesten Monoms in $\Psi$: $\#(\Psi) \geq 2^{n-k^*}$
\[
  U_\Psi^{\text{blind}} = \left\{ u\,|\,u \text{ ist eine Belegung der
Variablen} \right\}; \quad \left|U_\Psi^{\text{blind}}\right| = 2^n \]
\[
  \xi_{\text{blind}} = \frac{\left|U_\Psi^{\text{blind}}\right|}{\#(\Psi)} \leq
  \frac{2^n}{2^{n-k^*}} = 2^{k^*}
\]
\[
  \mathcal{A}_1(\Psi) = \frac{2^n}{\sqrt{2^{k^*}}} = 2^{n-k^*/2}
\]
\[
  \Psi_{\text{bad}} = x_1 \wedge x_2 \wedge \dots \wedge x_{n/2} \quad \quad
  k^* = \frac{n}{2} \quad \quad \#(\Psi_{\text{bad}}) = 2^{n/2} \quad \quad
  \xi_{\text{blind}} = 2^{n/2}
\]
\[
  \mathcal{A}_1(\Psi_{\text{bad}}) = 2^{\frac{3}{4}n}, \quad \text{Abweichung:
}\,2^{n/4}
\]
\begin{eqnarray*}
	S(\Psi) = & \bigcup_{j=1}^{m} \left\{ u\,|\,u \text{ erfüllt } C_j
	\right\} & = \bigcup_{j=1}^{m} \left\{ u\,|\,u \text{ erfüllt } C_j,
	\text{ aber kein } C_k \text{ mit } k < j \right\} \\
	S'(\Psi) = & \dots & = \bigcup_{j=1}^{m} \left\{ (u, j)\,|\,u \text{ erfüllt } C_j,
	\text{ aber kein } C_k \text{ mit } k < j \right\}
\end{eqnarray*}
\[ \#(\Psi) = |S(\Psi)| = |S'(\Psi)| \]
\[ U_{\Psi}^{\text{clever}} = \left\{ u\,|\,u \text{ macht } C_j \text{ wahr} \right\} =
\bigcupdot_{j=1}^{m} \left\{ (u, j)\,|\,u \text{ macht } C_j \text{ wahr} \right\} \]

\hfill 14. Juli
\vspace{0.5cm}

% TODO Bild fehlt!

$|U_I|$ bekannt, $|S(I)| = \#(I)$ gesucht, $\xi = \frac{|U_I|}{\#(I)}$
Expansion. Wichtig: \glqq gute Wahl\grqq\ von $U_I$

\begin{lemm}[Expansionslemma]
	\[
	  \xi_{\text{clever}} =
	  \frac{\left|U_{\Psi}^{\text{clever}}\right|}{\#(I)} \leq m
	\]
\end{lemm}
\begin{proof}
	Eine Belegung, die $\Psi$ erfüllt, kommt in $U_{\Psi}^{\text{clever}}$
	bis zu $m$ Mal vor, in $S(\Psi)$ aber genau einmal.
\end{proof}

$\mathcal{A}_2(\Psi) = \frac{1}{\sqrt{m}} \sum_{j=1}^m 2^{n-k_j}$ approximiert
$\#(\Psi)$ mit relativer Gütegarantie $\sqrt{m}$.
